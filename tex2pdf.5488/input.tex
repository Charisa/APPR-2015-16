\documentclass[]{article}
\usepackage{lmodern}
\usepackage{amssymb,amsmath}
\usepackage{ifxetex,ifluatex}
\usepackage{fixltx2e} % provides \textsubscript
\ifnum 0\ifxetex 1\fi\ifluatex 1\fi=0 % if pdftex
  \usepackage[T1]{fontenc}
  \usepackage[utf8]{inputenc}
\else % if luatex or xelatex
  \ifxetex
    \usepackage{mathspec}
    \usepackage{xltxtra,xunicode}
  \else
    \usepackage{fontspec}
  \fi
  \defaultfontfeatures{Mapping=tex-text,Scale=MatchLowercase}
  \newcommand{\euro}{€}
\fi
% use upquote if available, for straight quotes in verbatim environments
\IfFileExists{upquote.sty}{\usepackage{upquote}}{}
% use microtype if available
\IfFileExists{microtype.sty}{%
\usepackage{microtype}
\UseMicrotypeSet[protrusion]{basicmath} % disable protrusion for tt fonts
}{}
\usepackage[margin=1in]{geometry}
\usepackage{natbib}
\bibliographystyle{plainnat}
\usepackage{longtable,booktabs}
\usepackage{graphicx}
\makeatletter
\def\maxwidth{\ifdim\Gin@nat@width>\linewidth\linewidth\else\Gin@nat@width\fi}
\def\maxheight{\ifdim\Gin@nat@height>\textheight\textheight\else\Gin@nat@height\fi}
\makeatother
% Scale images if necessary, so that they will not overflow the page
% margins by default, and it is still possible to overwrite the defaults
% using explicit options in \includegraphics[width, height, ...]{}
\setkeys{Gin}{width=\maxwidth,height=\maxheight,keepaspectratio}
\ifxetex
  \usepackage[setpagesize=false, % page size defined by xetex
              unicode=false, % unicode breaks when used with xetex
              xetex]{hyperref}
\else
  \usepackage[unicode=true]{hyperref}
\fi
\hypersetup{breaklinks=true,
            bookmarks=true,
            pdfauthor={Sara Korat},
            pdftitle={Poročilo pri predmetu Analiza podatkov s programom R},
            colorlinks=true,
            citecolor=blue,
            urlcolor=blue,
            linkcolor=magenta,
            pdfborder={0 0 0}}
\urlstyle{same}  % don't use monospace font for urls
\setlength{\parindent}{0pt}
\setlength{\parskip}{6pt plus 2pt minus 1pt}
\setlength{\emergencystretch}{3em}  % prevent overfull lines
\setcounter{secnumdepth}{0}

%%% Use protect on footnotes to avoid problems with footnotes in titles
\let\rmarkdownfootnote\footnote%
\def\footnote{\protect\rmarkdownfootnote}

%%% Change title format to be more compact
\usepackage{titling}

% Create subtitle command for use in maketitle
\newcommand{\subtitle}[1]{
  \posttitle{
    \begin{center}\large#1\end{center}
    }
}

\setlength{\droptitle}{-2em}
  \title{Poročilo pri predmetu Analiza podatkov s programom R}
  \pretitle{\vspace{\droptitle}\centering\huge}
  \posttitle{\par}
  \author{Sara Korat}
  \preauthor{\centering\large\emph}
  \postauthor{\par}
  \date{}
  \predate{}\postdate{}

\usepackage[slovene]{babel}
\usepackage{graphicx}


\begin{document}

\maketitle


\begin{verbatim}
## Warning: package 'knitr' was built under R version 3.2.3
\end{verbatim}

\begin{verbatim}
## Warning: package 'XML' was built under R version 3.2.3
\end{verbatim}

\begin{verbatim}
## Warning: package 'RCurl' was built under R version 3.2.3
\end{verbatim}

\begin{verbatim}
## Warning: package 'bitops' was built under R version 3.2.3
\end{verbatim}

\begin{verbatim}
## Warning: package 'ggplot2' was built under R version 3.2.3
\end{verbatim}

\begin{verbatim}
## Warning: package 'plyr' was built under R version 3.2.3
\end{verbatim}

\begin{verbatim}
## Warning: package 'dplyr' was built under R version 3.2.3
\end{verbatim}

\begin{verbatim}
## Warning: package 'reshape2' was built under R version 3.2.3
\end{verbatim}

\begin{verbatim}
## Warning: package 'httr' was built under R version 3.2.3
\end{verbatim}

\begin{verbatim}
## Warning: package 'sp' was built under R version 3.2.3
\end{verbatim}

\begin{verbatim}
## Warning: package 'maptools' was built under R version 3.2.3
\end{verbatim}

\begin{verbatim}
## Warning: package 'digest' was built under R version 3.2.3
\end{verbatim}

\section{Izbira teme}\label{izbira-teme}

Indeks cen življenjskih potrebščin v Sloveniji v letošnjem letu (2015).
Primerjava z ostalimi evropskimi državami (3. faza). Indeks cen
življenjskih potrebščin meri spremembe cen izdelkov in storitev glede na
sestavo izdatkov, ki jih domače prebivalstvo namenja za nakupe predmetov
končne porabe doma in v tujini.

\begin{center}\rule{0.5\linewidth}{\linethickness}\end{center}

\section{Obdelava, uvoz in čiščenje
podatkov}\label{obdelava-uvoz-in-ciscenje-podatkov}

Uvozila sem podatke o indeksih življenjskih potrebščin v Sloveniji v
obliki .CSV s Statističnega urada Slovenije ter v obliki HTML iz spletne
strani
(\url{http://www.numbeo.com/cost-of-living/rankings_by_country.jsp?title=2015-mid\&region=150}).

Razpredelnica s slovenskimi podatki prikazuje različne indekse cen:
indeks: tekoči mesec / prejšnji mesec, indeks: tekoči mesec / isti mesec
prejšnjega leta, indeks: povprečje mesecev tekočega leta / povprečje
istih mesecev prejšnjega leta (spremembe cen od začetka leta do tekočega
meseca gleda na enako obdobje). Tabela vsebuje 128 vrstic. (Vsak četrti
stolpec iz originalne tabele, ki je prikazoval povprečno 12-mesečno
rast, sem izbrisala.)

\begin{longtable}[c]{@{}lrrrrrrrrrrrrrrrrrrrrrrrrrrrrrrrrr@{}}
\toprule
& januar2015.december2014 & januar2015.januar2014 &
povprečje\_januar\_2015.povprečje\_januar\_2014 & februar2015.januar2015
& februar2015.februar2014 &
povprečje\_januar,februar\_2015.povprečje\_januar,februar\_2014 &
marec2015.februar2015 & marec2015.marec2014 &
povprečje\_januar-marec\_2015.povprečje\_januar-marec\_2014 &
april2015.marec2015 & april2015.april2014 &
povprečje\_januar-april\_2015.povprečje\_januar-april\_2014 &
maj2015.april2015 & maj2015.maj2014 &
povprečje\_januar-maj\_2015.povprečje\_januar-maj\_2014 &
junij2015.maj2015 & junij2015.junij2014 &
povprečje\_januar-junij\_2015.povprečje\_januar-junij\_2014 &
julij2015.junij2015 & julij2015.julij2014 &
povprečje\_januar-julij\_2015.povprečje\_januar-julij\_2014 &
avgust2015.julij2015 & avgust2015.avgust2014 &
povprečje\_januar-avgust\_2015.povprečje\_januar-avgust\_2014 &
september2015.avgust2015 & september2015.september2014 &
povprečje\_januar-september\_2015.povprečje\_januar-september\_2014 &
oktober2015.september2015 & oktober2015.oktober2014 &
povprečje\_januar-oktober\_2015.povprečje\_januar-oktober\_2014 &
november2015.oktober2015 & november2015.oktober2014 &
povprečje\_januar-november\_2015.povprečje\_januar-november\_2014\tabularnewline
\midrule
\endhead
Hrana & 101.4 & 98.8 & 98.8 & 100.1 & 99.7 & 99.2 & 100.5 & 100.7 & 99.7
& 100.5 & 100.8 & 100.0 & 101.8 & 101.3 & 100.2 & 98.9 & 101.2 & 100.4 &
99.3 & 101.4 & 100.5 & 99.6 & 101.8 & 100.7 & 100.3 & 101.1 & 100.7 &
99.3 & 101.0 & 100.8 & 99.9 & 101.0 & 100.8\tabularnewline
Kruh in drugi izdelki iz žit & 100.8 & 99.0 & 99.0 & 98.1 & 98.1 & 98.5
& 102.1 & 99.9 & 99.0 & 98.7 & 99.1 & 99.0 & 100.4 & 99.0 & 99.0 & 99.7
& 98.7 & 99.0 & 100.1 & 98.7 & 98.9 & 100.6 & 99.2 & 99.0 & 100.0 & 99.7
& 99.1 & 99.8 & 99.9 & 99.1 & 100.4 & 100.2 & 99.2\tabularnewline
Meso & 99.8 & 98.9 & 98.9 & 99.4 & 97.9 & 98.4 & 100.0 & 98.8 & 98.5 &
99.9 & 99.1 & 98.7 & 99.9 & 98.7 & 98.7 & 100.1 & 98.7 & 98.7 & 101.3 &
100.4 & 98.9 & 99.6 & 99.9 & 99.0 & 100.1 & 100.1 & 99.2 & 99.3 & 100.0
& 99.2 & 100.1 & 99.1 & 99.2\tabularnewline
Ribe & 102.2 & 99.8 & 99.8 & 99.2 & 98.0 & 98.9 & 100.2 & 100.3 & 99.4 &
99.8 & 99.7 & 99.4 & 101.7 & 100.4 & 99.6 & 99.8 & 101.4 & 99.9 & 98.7 &
100.2 & 99.9 & 101.5 & 102.5 & 100.3 & 100.6 & 103.8 & 100.6 & 98.5 &
103.7 & 100.9 & 100.5 & 103.2 & 101.1\tabularnewline
Mleko, mlečni izdelki in jajca & 99.6 & 98.9 & 98.9 & 99.6 & 99.6 & 99.2
& 100.1 & 99.5 & 99.3 & 99.9 & 99.5 & 99.4 & 99.7 & 98.4 & 99.2 & 100.9
& 99.1 & 99.2 & 99.3 & 98.6 & 99.1 & 100.3 & 99.0 & 99.1 & 100.1 & 99.1
& 99.1 & 99.5 & 99.1 & 99.1 & 100.4 & 98.9 & 99.1\tabularnewline
Olje in maščoba & 101.6 & 100.3 & 100.3 & 98.4 & 98.8 & 99.5 & 102.6 &
101.0 & 100.0 & 98.4 & 100.0 & 100.0 & 101.2 & 101.1 & 100.2 & 99.5 &
100.1 & 100.2 & 100.0 & 99.7 & 100.1 & 100.7 & 101.2 & 100.3 & 100.3 &
101.4 & 100.4 & 98.9 & 101.2 & 100.5 & 101.1 & 102.3 &
100.6\tabularnewline
\bottomrule
\end{longtable}

Tabela osnovne\_dobrine prikazuje le podatke indeksov, izračunane na
podlagi tekočega in prejšnjega meseca (11 stolpcev). Izbrala sem tudi
nekaj najosnovnejših dobrin/storitev (13 vrstic).

\begin{longtable}[c]{@{}lrrrrrrrrrrr@{}}
\toprule
& januar2015.december2014 & februar2015.januar2015 &
marec2015.februar2015 & april2015.marec2015 & maj2015.april2015 &
junij2015.maj2015 & julij2015.junij2015 & avgust2015.julij2015 &
september2015.avgust2015 & oktober2015.september2015 &
november2015.oktober2015\tabularnewline
\midrule
\endhead
Hrana & 101.9 & 100.0 & 100.0 & 100.0 & 100.0 & 100.0 & 100.0 & 100.0 &
100.0 & 100.0 & 100.0\tabularnewline
Obleka in storitve za obleko & 101.4 & 100.1 & 100.5 & 100.5 & 101.8 &
98.9 & 99.3 & 99.6 & 100.3 & 99.3 & 99.9\tabularnewline
Stanovanje & 101.2 & 100.0 & 100.0 & 100.0 & 100.0 & 100.2 & 100.0 &
100.0 & 100.1 & 100.2 & 100.0\tabularnewline
Komunalne in druge storitve & 100.3 & 100.0 & 100.1 & 100.0 & 99.8 &
100.0 & 100.0 & 100.0 & 100.0 & 100.0 & 100.1\tabularnewline
Goriva in energija & 100.1 & 103.5 & 99.3 & 99.2 & 100.0 & 99.2 & 104.5
& 100.1 & 99.5 & 100.8 & 100.0\tabularnewline
Zdravje & 100.1 & 99.4 & 100.3 & 100.1 & 100.0 & 100.2 & 100.6 & 100.2 &
100.0 & 99.8 & 100.0\tabularnewline
\bottomrule
\end{longtable}

Naredila sem novo tabelo (osnovne\_dobrine\_graf), ki vsebuje le nekaj
informacij (4 dobrine) iz tabele osnovne\_dobrine (prikazane spodaj), s
pomočjo katere sem dobila podatke za graf. Dobrine/storitve sem izbrala
na podlagi vrednosti indeksov le-teh (dve največji in najmanjši
vrednosti v prvem indeksu). Za vrednost po stolcih sem vzela prva dva in
zadnja dva indeksa v letu.

\begin{figure}

{\centering \includegraphics{projekt_files/figure-latex/stolpicni_diagram-1} 

}

\caption{Stolpični diagram indeksov cen dveh največjih in dveh najmanjši vrednosti indeksa.}\label{fig:stolpicni_diagram}
\end{figure}

Tabela (prvotno HTML) je v tej fazi bolj kot ne nedotaknjena. Vsebuje 40
vrstic in 6 stolpcev. Vrstice označujejo imena držav, stolpci pa
različne indekse (indeks cen življenjskih potrebščin, indeks najemnin,
indeks vsote življenjskih potrebščin in najemnin, indeks hrane in
pijače, indeks cen restavracijskih ponudb, indeks lokalne moči nakupa).

\begin{longtable}[c]{@{}lrrrrrr@{}}
\toprule
& Consumer Price Index & Rent Index & Consumer Price Plus Rent Index &
Groceries Index & Restaurant Price Index & Local Purchasing Power
Index\tabularnewline
\midrule
\endhead
Switzerland & 124.51 & 54.53 & 88.82 & 114.81 & 137.82 &
210.00\tabularnewline
Norway & 109.30 & 40.41 & 74.17 & 95.90 & 136.81 & 139.78\tabularnewline
Iceland & 95.41 & 29.04 & 61.57 & 88.13 & 113.07 & 111.09\tabularnewline
Denmark & 88.31 & 26.33 & 56.70 & 71.67 & 112.78 & 164.26\tabularnewline
United Kingdom & 86.68 & 33.50 & 59.56 & 75.10 & 96.69 &
133.64\tabularnewline
Luxembourg & 80.81 & 50.91 & 65.56 & 65.30 & 104.96 &
153.61\tabularnewline
\bottomrule
\end{longtable}

V 3. fazi bom analizirala spremembe indeksov določenih dobrin, med
katerimi bodo najbolj zanimive ravno tiste z največjimi razlikami v
vrednosti indeksa. Seveda pa bom podatke primerjala s podatki drugih
evropskih držav. (Za osnovo planiram sicer izbrati tabelo evropskih
indeksov (na voljo manj podatkov iz tujine kot pa iz Slovenije), le-te
pa bom primerjala z indeksi slovenskih dobrin zadnjega meseca.)
Primerjala bom tudi podatke ostalih držav med sabo, kakšne so največje
razlike v cenah. Predvidevam, da imajo revnejše države nižje indekse od
bogatejših.

\begin{center}\rule{0.5\linewidth}{\linethickness}\end{center}

\end{document}
